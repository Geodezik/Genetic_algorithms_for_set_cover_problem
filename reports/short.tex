\documentclass[
	11pt,
]{beamer}

\graphicspath{{Images/}{./}}
\usetheme{Copenhagen}
\usefonttheme{default}
\useinnertheme{circles}
\usepackage{palatino}
\usepackage[default]{opensans}
\usepackage[english,russian]{babel}
\usepackage{booktabs}
\usepackage{graphicx}
\usepackage{caption}
\usepackage{subcaption}

\title[Генетический подход к выбору частичных порядков]{Генетический выбор частичных порядков на множестве значений признаков в задаче классификации}
\author[Сорокин Олег, 317 группа ММП ВМК МГУ]{Сорокин Олег, 317}
\institute[ММП ВМК МГУ]{ММП ВМК МГУ}
\date[\today]{Спецсеминар \\ \today}

\begin{document}

\begin{frame}
	\titlepage
\end{frame}

\begin{frame}
	\tableofcontents
\end{frame}

\section{Обзор статей про частичные порядки}

\subsection{Общая постановка задачи}

\begin{frame}
	\frametitle{Постановка задачи классификации для произведений частичных порядков}
	
    Рассматривается более общая постановка задачи, приведённая ниже.

    \bigskip

	$M = \cup_{n=1}^{l}K_n$, где $K_i \cap K_j = \varnothing$ при $i \neq j$.

	\bigskip

	Пусть $M$ представимо в виде $N_1 \times ... \times N_n$, где $N_i$ ($i \in \{1, 2, ..., n\}$) — конечное множество допустимых значений признака $x_i$. Не ограничивая общности, можно считать, что $N_i$ имеет наибольший элемент $k_i$.

	Пусть также задан набор прецедентов $S_1 = (a_{11}, ..., a_{1n}), S_2 = (a_{21}, ..., a_{2n}), ..., S_m = (a_{m1}, ..., a_{mn})$.

	\bigskip

	Требуется по предъявленному набору значений признаков $(a_1, ..., a_n)$ объекта $S \in M$ (класс которого, вообще говоря, неизвестен) определить этот класс.

\end{frame}

\subsection{Алгоритм классификации и процедуры упорядочения}

\begin{frame}
	\frametitle{О рассматриваемом классе алгоритмов}
	
	\begin{enumerate}
		\item Обучение: для каждого класса $K$ строится некоторое множество представительных эл. кл. $C^A(K)$.
		\item Процедура голосования: вычисление оценок вида
			  $$\Gamma(S, K) = \frac{1}{|C^A(K)|} \sum_{(\sigma, H) \in C^A(K)} P_{(\sigma, H)} * \hat{B}(\sigma, S, H)$$
	\end{enumerate}
\end{frame}

\begin{frame}
	\frametitle{Быстрая процедура независимого линейного упорядочения значений признаков}
    Частичные порядки в этой процедуре строятся после анализа частот встречаемости значений признаков.

	\begin{block}{Определение}
		Частичный порядок на $M$ называется $(A, K)$-корректным, если алгоритм $A$ правильно классифицирует каждый объект из $R(K)$.
	\end{block}

	\begin{exampleblock}{Замечания}
        \begin{enumerate}
		    \item Порядок на множестве значений каждого признака выбирается независимо от выбора порядков для других признаков.
		    \item Описанная процедура не является корректной в смысле определения, приведённого выше.
        \end{enumerate}
	\end{exampleblock}
\end{frame}

\begin{frame}
	\frametitle{Процедура корректного упорядочения значений признаков}
    В ходе процедуры строится булева матрица $B_K$ особого вида.

    \bigskip

    Рассматривается некоторый алгоритм $A$ из описанного ранее класса. Для него справедлива
	\begin{block}{Теорема.}
		Частичный порядок, заданный на множестве $M$, является $(A, K)$-корректным тогда и только тогда, когда существует неприводимое покрытие $H$ матрицы $B_K$ такое, что $\forall j \in \{1, 2, ..., n\}$ и $\forall a, b \in N_j$ ($a < b$) столбец $(j, b, a)$ не входит в $H$.
	\end{block}
\end{frame}

\end{document}